

%% La classe stageM2R s'appuie sur la classe memoir, plus d'information sur le paquet: http://www.ctan.org/pkg/memoir
%% option possible de la classe stageM2R
% utf8  -> encodage du texte UTF8 (défaut: Latin1)
% final -> mode rapport de stage final (défaut: mode étude bibliographique)
% private -> indique une soutenance privée (défaut: soutenance publique)
\documentclass[utf8]{stageM2R} %-> etude bibliographique
%\documentclass[utf8,final]{stageM2R} %-> rapport final


\usepackage{lipsum} %% a supprimer 

%%%%%%%%%%%%%%%%%%%%%%%%%%%%
%%% Déclaration du stage %%%
%%%%%%%%%%%%%%%%%%%%%%%%%%%%

%% auteur
\author{Prénom Nom}
%% encadrants
\supervisors{Prénom1 Nom1\\Prénom2 Nom2}
%% lieu du stage (Optionnel)
\location{LIRMM UM5506 - CNRS, Université de Montpellier}
%% titre du stage
\title{Intitulé du stage} 
%% parcours du master
\track{monparcours}  
%% date de soutenance (Optionnel)
\date{\today} 
%% version du rapport (Optionnel)
\version{1}
%% Résumé en francais
\abstractfr{
Ce stage de master.
}
%% Résumé en anglais
\abstracteng{
  This master thesis.
}



\begin{document}   
%\selectlanguage{english} %% --> turn the document into english mode (Default is french)
\selectlanguage{french} 
\frontmatter  %% -> pas de numérotation numérique
\maketitle    %% -> création de la page de garde et des résumés
\cleardoublepage   
\tableofcontents %% -> table des matières
\mainmatter  %% -> numérotation numérique


%%%%%%%%%%%%%%%%%%%%%%%%%%%%%%
%%%%    DEBUT DU RAPPORT  %%%%
%%%%%%%%%%%%%%%%%%%%%%%%%%%%%%

\chapter{Introduction}
\lipsum
\chapter{Lorem Ipsum}
\lipsum
\section{Lorem Ipsum} 
\lipsum
\section{Lorem Ipsum} 
\lipsum    
\end{document}

    
%%% Local Variables: 
%%% mode: latex
%%% TeX-master: t
%%% End:  
 